\chapter{Аналитическая часть} \label{chapt1}

\section{Типы пропусков} \label{sect1_1}

Согласно [] выделяют следующие типы пропусков:
\begin{enumerate}
  \item Полностью случайные (missing completely at random,
    MCAR). Вероятность отсутствия не зависит ни от значения самого
    измеряемого признака, ни от значений других признаков. Примером
    возникновения пропусков данного типа является потеря образца крови
    пацента, следствием которой является невозможностьь измерения значений
    ряда признаков.
  \item Случайные (missing at random; MAR). Вероятность отсутствия не
    зависит от значений самого измеряемого признака, но обусловлена
    значениями других признаков. В качестве примера можно привести
    противопоказание некоторых анализов пацентам вследствие ряда
    заболеваний, при которых проведение данных анализов может нанести вред
    организму.
  \item Неслучайные (not missing at random; NMAR). Вероятность
    отсутствия зависит от значения измеряемого признака. Например, если
    измеряемое значение не попадает в диапозон чувствительности
    измерительного прибора, то данный пропуск принято называть случайным.
\end{enumerate}
