
{\actuality} С помощью методов машинного обучения в настоящее время
решаются задачи во многих областях человеческой деятельности, а
именно:
\begin{enumerate}
  \item Задача медицинской диагностики, при решении которой по
    признаковому описанию пациента (результаты обследований, пол, возраст,
    наличие болей, тяжесть состояния) необходимо поставить ему диагноз и
    назначить соответствующее лечение.
  \item Задача определения социальной группы респондента по
    результатам опроса. Признаками в данном случае могут быть: возраст,
    доход, количество детей и~т.\:д.
  \item Задача определения вредоносного сетевого трафика с целью
    обнаружения вторжений в компьютерные сети. Для анализа из трафика
    могут извлекаться следующие признаки: протокол: IP-опции,
    характеристики фрагментации и т.д.
\end{enumerate}

Широкий класс задач, как правило, характеризуется неполнотой исходных
данных, например~\cite{garcia2009pattern}:
\begin{enumerate}
  \item При решении задач медицинской диагностики результаты некоторых
    aнализов могут отсутствовать; история болезни может быть неполной,
    если пациент без сознания~\cite{garcia2009pattern,liu2005analysis}.
  \item В задачах социального анализа пропуски в исходных данных могут
    быть следствием нежелания респондентов отвечать на сензитивные
    вопросы~\cite{liu2005analysis,zagnieva2008solving,rubin1987multiple}.
  \item Часть данных, собираемых во время мониторинга трафика
    компьютерных сетей, может отсутствовать из-за сбоев оборудования или
    программного обеспечения, осуществляющих мониторинг~\cite{garcia2009pattern}.
\end{enumerate}

% Обзор, введение в тему, обозначение места данной работы в
% мировых исследованиях.

% \ifsynopsis
% Этот абзац появляется только в~автореферате.
% Для формирования блоков, которые будут обрабатываться только в~автореферате,
% заведена проверка условия \verb!\!\verb!ifsynopsis!.
% Значение условия задаётся в~основном файле документа (\verb!synopsis.tex! для
% автореферата).
% \else
% Этот абзац появляется только в~диссертации.
% Через проверку условия \verb!\!\verb!ifsynopsis!, задаваемого в~основном файле
% документа (\verb!dissertation.tex! для диссертации), можно сделать новую
% команду, обеспечивающую появление цитаты в~диссертации, но~не~в~автореферате.
% \fi

% {\progress} 
% Этот раздел должен быть отдельным структурным элементом по
% ГОСТ, но он, как правило, включается в описание актуальности
% темы. Нужен он отдельным структурынм элемементом или нет ---
% смотрите другие диссертации вашего совета, скорее всего не нужен.

{\aim} данной работы является разработка метода заполнения
отсутствующих данных, обеспечивающего приемлемое качество заполнения
при решении некоторого класса задач.

Для~достижения поставленной цели необходимо было решить следующие {\tasks}:
\begin{enumerate}
  \item Провести анализ современных методов, используемых для решения
    задач машинного обучения в условиях неполноты данных.
  \item Разработать метод заполнения отсутствующих данных и исследовать
    его алгоритмическую реализацию.
  \item Реализовать разработанное алгоритмическое обеспечения метода
    программно.
  \item Провести численные эксперименты, исследующие программную
    реализацию, с целью определения границ применимости разработанного
    метода.
  \item Исследовать, разработать, вычислить и~т.\:д. и~т.\:п.
\end{enumerate}


{\novelty}
\begin{enumerate}
  \item Впервые \ldots
  \item Впервые \ldots
  \item Было выполнено оригинальное исследование \ldots
\end{enumerate}

{\influence} \ldots

{\methods} \ldots

{\defpositions}
\begin{enumerate}
  \item Первое положение
  \item Второе положение
  \item Третье положение
  \item Четвертое положение
\end{enumerate}

{\reliability} полученных результатов обеспечивается \ldots \ Результаты находятся в соответствии с результатами, полученными другими авторами.


{\probation}
Основные результаты работы докладывались~на:
перечисление основных конференций, симпозиумов и~т.\:п.

{\contribution} Автор принимал активное участие \ldots

%\publications\ Основные результаты по теме диссертации изложены в ХХ печатных изданиях~\cite{Sokolov,Gaidaenko,Lermontov,Management},
%Х из которых изданы в журналах, рекомендованных ВАК~\cite{Sokolov,Gaidaenko}, 
%ХХ --- в тезисах докладов~\cite{Lermontov,Management}.

\ifnumequal{\value{bibliosel}}{0}{% Встроенная реализация с загрузкой файла через движок bibtex8
    \publications\ Основные результаты по теме диссертации изложены в XX печатных изданиях, 
    X из которых изданы в журналах, рекомендованных ВАК, 
    X "--- в тезисах докладов.%
}{% Реализация пакетом biblatex через движок biber
%Сделана отдельная секция, чтобы не отображались в списке цитированных материалов
    \begin{refsection}[vak,papers,conf]% Подсчет и нумерация авторских работ. Засчитываются только те, которые были прописаны внутри \nocite{}.
        %Чтобы сменить порядок разделов в сгрупированном списке литературы необходимо перетасовать следующие три строчки, а также команды в разделе \newcommand*{\insertbiblioauthorgrouped} в файле biblio/biblatex.tex
        \printbibliography[heading=countauthorvak, env=countauthorvak, keyword=biblioauthorvak, section=1]%
        \printbibliography[heading=countauthorconf, env=countauthorconf, keyword=biblioauthorconf, section=1]%
        \printbibliography[heading=countauthornotvak, env=countauthornotvak, keyword=biblioauthornotvak, section=1]%
        \printbibliography[heading=countauthor, env=countauthor, keyword=biblioauthor, section=1]%
        \nocite{%Порядок перечисления в этом блоке определяет порядок вывода в списке публикаций автора
                vakbib1,vakbib2,%
                confbib1,confbib2,%
                bib1,bib2,%
        }%
        \publications\ Основные результаты по теме диссертации изложены в~\arabic{citeauthor}~печатных изданиях, 
        \arabic{citeauthorvak} из которых изданы в журналах, рекомендованных ВАК, 
        \arabic{citeauthorconf} "--- в~тезисах докладов.
    \end{refsection}
    \begin{refsection}[vak,papers,conf]%Блок, позволяющий отобрать из всех работ автора наиболее значимые, и только их вывести в автореферате, но считать в блоке выше общее число работ
        \printbibliography[heading=countauthorvak, env=countauthorvak, keyword=biblioauthorvak, section=2]%
        \printbibliography[heading=countauthornotvak, env=countauthornotvak, keyword=biblioauthornotvak, section=2]%
        \printbibliography[heading=countauthorconf, env=countauthorconf, keyword=biblioauthorconf, section=2]%
        \printbibliography[heading=countauthor, env=countauthor, keyword=biblioauthor, section=2]%
        \nocite{vakbib2}%vak
        \nocite{bib1}%notvak
        \nocite{confbib1}%conf
    \end{refsection}
}